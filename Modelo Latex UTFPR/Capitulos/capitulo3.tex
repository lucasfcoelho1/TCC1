%%%%%%%%%%%%%%%%%%%%%%%%%%%%%%%%%%%%%%%%%%%%%%%%%%%%%%%%%%%%%%%%%%%%%%%%%%%%%%%
% CAP�TULO 3
\chapter{Proposta}  

Com base no conte�do apresentado no \chaptername 2, o presente trabalho prop�e verificar a efic�cia da ferramenta x-PATeSco, com o intuito de analisar se os \textit{scripts} de testes gerados conseguem se adaptar a mudan�as entre as vers�es de uma mesma \textit{app}.

Para este estudo foram selecionadas tr�s \textit{apps} desenvolvidas em Xamarin. Elas ser�o identificadas por \textit{APP1}, \textit{APP2} e \textit{APP3}. No in�cio do estudo ser� utilizada o apk da primeira vers�o publicada de cada \textit{app}. Ela ser� instalada num dispositivo com SO Android e ser� submetida a ferramenta x-PATeSco. Na ferramenta ser�o configurados os casos de teste, que depois ser�o exportados como um projeto para o Visual Studio. Feito isso, o projeto de teste ser� aplicado em cada vers�o posterior de cada \textit{app}, verificando a efic�cia da ferramenta em lidar com as altera��es de interface. A \textit{APP1} possui vite e oito vers�es publicadas na Play Store, a \textit{APP2} possui cinco e a \textit{APP3} possui seis. As informa��es das \textit{apps} podem ser vistas na \figurename.

%figura apps



O cronograma compreende o per�odo de agosto de 2017 a junho de 2018.
\\*\textbf{Atividade 1:} Defini��o do tema do trabalho.
\\*\textbf{Atividade 2:} Estudar o \textit{framework} Xamarin, utilizado para cria��o das \textit{apps}.
\\*\textbf{Atividade 3:} Estudar a ferramenta x-PATeSco que ser� utilizado para gerar os \textit{scripts} de teste.
\\*\textbf{Atividade 4:} Estudar o \textit{framework} Appium que ser� o servidor para os testes nos dispositivos.
\\*\textbf{Atividade 5:} Buscar trabalhos relacionados que apresentam alguma solu��o para realizar testes automatizados em \textit{apps cross-platform}.
\\*\textbf{Atividade 6:} Entrega e apresenta��o da proposta. 
\\*\textbf{Atividade 7:} Realizar corre��es na proposta sugeridas pela banca.
\\*\textbf{Atividade 8:} Elabora��o do trabalho com apresenta��o dos resultados obtidos.
\\*\textbf{Atividade 9:} Entrega e apresenta��o do trabalho final para a banca examinadora.
\\*\textbf{Atividade 10:} Corre��es no trabalho final conforme sugest�es da banca. 


