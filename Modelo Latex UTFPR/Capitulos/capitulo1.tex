%%%%%%%%%%%%%%%%%%%%%%%%%%%%%%%%%%%%%%%%%%%%%%%%%%%%%%%%%%%%%%%%%%%%%%%%%%%%%%%
% CAP�TULO 1
\chapter{Introdu��o}  

Atualmente os dispositivos m�veis s�o parte da vida das pessoas e est�o dispon�veis nos mais diferentes formatos, como os tablets, smartphones e werables. Esses dispositivos port�teis possuem alta capacidade de processamento, muito espa�o de armazenamento e uma variedade diversificada de sensores \cite{7372032}. Esses dispositivos s�o controlados por um sistema operacional (SO), do qual � a interface que o usu�rio utiliza. De acrodo com a pesquisa da International Data Corporation \cite{IDC2017} sobre o Market Share dos SOs por dispositivo do primeiro trimestre de 2017(1Q17), os sistemas mais utilizados s�o o Android OS [3] com 85.0\% e Apple iOS [4] com 14.7\%. Cada SO possui uma loja de aplicativos (\textit{app stores}) onde os desenvolvedores podem submeter seus apps e usu�rios podem fazer download e instala-los em seus dispositivos. Numa an�lise dessas app stores feita pela appfigures \cite{Chantelle2017}, indicou que a quantidade de apps ativos em 2016 na loja oficial do Android, a Google Play [33], contava com aproximadamente 2.81 milh�es de aplicativos e a iOS App Store [34], da Apple, tinha algo em torno de 2.26 milh�es. Esse aplicativos s�o utilizados em todas as �reas, finan�as, educa��o, jogos


\section{Motiva��o}



\section{Objetivos}

\subsection{Objetivo Geral}



\subsection{Objetivos Espec\'ificos}
\begin{enumerate}
\item dfsdsfd
\item fwfsdfsd
\end{enumerate}


\begin{itemize}
	\item Obter documentos acad\^emicos automaticamente formatados com corre\c{c}\~ao e perfei\c{c}\~ao est\'etica.
	\item Desonerar autores da tediosa tarefa de formatar documentos acad\^emicos, permitindo sua concentra\c{c}\~ao no conte\'udo do mesmo.
	\item Desonerar orientadores e examinadores da tediosa tarefa de conferir a formata\c{c}\~ao de documentos acad\^emicos, permitindo sua concentra\c{c}\~ao no conte\'udo do mesmo.
\end{itemize}

\section{Organiza��o do Texto}

